%!TEX program = xelatex

% Pass 'debug' option to show boxes
\documentclass[a4paper]{genard-cv}
\usepackage[french]{babel}

% Information
\renewcommand{\firstname}{Elie}
\renewcommand{\lastname}{Génard}
\renewcommand{\subtitle}{CV pour un poste d'ingénieur créatif}
\renewcommand{\phone}{06 xx xx xx xx}
\renewcommand{\email}{xxxxxxxxxx@xxxx.xxx}
\renewcommand{\adress}{xx xxx xxxxxxxxx}
\renewcommand{\city}{xxxxx xxxxxxx xxx xxxxx}
\renewcommand{\profession}{Ingénieur en physique et électronique}

\linespread{1.2}

\begin{document}

\begin{header}

	% \photo[shadow=true]{photo/photo.png}

	\begin{contact}[vcard=false]
		\addinfo{Email}{\href{mailto:\email}{\email}}
		\addinfo{Téléphone}{\phone}
		\addinfo{Adresse}{\adress \\ \city}
		% \addinfo{}{\city}
	\end{contact}

\end{header}

\begin{body}
	\experience
		\begin{entry}
			{Mars-Juillet 2014}{Stage dans un laboratoire d'informatique}{Bordeaux}
			{LaBRI - Laboratoire Bordelais de Recherche en Informatique}
			\textbf{Projet :} Création de résumés vidéo scalables de documents culturels en utilisant\\
			un cube de données (projet international franco-mexicain)\\
			~~~~Extraction et clustering de descripteurs de caractéristiques audiovisuels bas niveau\\
			~~~~Programmation en C++
		\end{entry}
		\begin{entry}
			{Juin-Septembre 2013}{Stage en tant qu'assistant ingénieur en \\traitement du signal}{Grenoble}
			{Movea}
			\textbf{Projet :} Détection de pas pour la navigation piétonne\\
			~~~~Analyse de données enregistrées par une centrale inertielle embarquée\\
			~~~~Modèle de détection de pas réalisé avec Simulink
		\end{entry}

	\education
		\begin{entry}
			{2011 - 2014}{Ecole d'ingénieur à Grenoble INP Phelma}{Grenoble}
			{PHELMA - Ecolde d'ingénieur en Physique, Electronique et Matériaux}
			\textit{2013-2014} Semestre Art, Science et Techonologie\\
			\textit{HMI - Interfaces haptiques - Programmation pour la création interactivve}\\
			\textbf{Projet :} Jonglage interactif avec un diabolo\\
			~~~~Tracking des bras du jongleur et du diabolo avec une Kinect\\
			~~~~Visualisation utilisant un système de particules qui réagit aux mouvements du jongleur\\
			~~~~Génération sonore réalisée avec SuperCollider\\
			~~~~{\itshape [~code disponible sur \href{https://github.com/elaye/AST\_diabolo\_particles}{https://github.com/elaye/AST\_diabolo\_particles}~]}
			\medskip\\

			\textit{2012-2013} Physique et nanosciences\\
			\textit{Physique quantique - Physique du solide - Physique des lasers}\\
			\textbf{Projet :} Mémoires MRAM ultrarapides\\
			~~~~Simulation du comportement de nanopilliers magnétiques
			\medskip\\

			\textit{2011-2012} Physique, Electronique et Télécoms\\
			\textit{Physique - Electronique - Traitement du signal}\\
			\textbf{Projet :} Compétition CanSat\\
			~~~~Electronique embarquée et programmation Arduino
		\end{entry}
		\begin{entry}
			{2008 - 2011}{Classe Préparatoire aux Grandes Ecoles}{Poitiers}
			{MPSI - MP - MP*}
			Physique et Mathématiques\\
			Option Sciences de l'Ingénieur
		\end{entry}
		\begin{entry}
			{2008}{Baccalauréat - Section S}{Barbezieux}
			{Mention Bien}
		\end{entry}

\end{body}

\begin{aside}

	\begin{about}
		\addinfo{Age:}{24}
		\addinfo{Français:}{langue maternelle}
		\addinfo{Anglais:}{courant}
	\end{about}\noindent

	\skills
	
	\begin{skillgroup}{Langages}
		\skill{C++}{3}
		\skill{Ruby}{3}
		\skill{Java}{1}
	\end{skillgroup}

	\begin{skillgroup}{Web}
		\skill{HTML5/CSS3}{3}
		% \skill{CSS3}{3}
		\skill{Javascript}{2}
		\skill{Ruby on Rails}{2}
		\skill{Node.js}{2}
	\end{skillgroup}

	\begin{skillgroup}{OS}
		\skill{GNU/Linux}{3}
		\skill{OS X}{2}
		\skill{Windows}{2}
	\end{skillgroup}

	\begin{skillgroup}{Misc}
		\skill{openFrameworks}{4}
		\skill{\LaTeX}{4}
		\skill{Photoshop}{3}
	\end{skillgroup}

	\skillscaption
\end{aside}

\end{document}
