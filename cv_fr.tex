%!TEX program = xelatex

% Pass 'debug' option to show boxes
\documentclass[a4paper]{genard-cv}
\usepackage[french]{babel}
\usepackage[utf8]{inputenc}

% Information
\renewcommand{\firstname}{Elie}
\renewcommand{\lastname}{Génard}
\renewcommand{\subtitle}{CV pour un poste de programmeur}
\renewcommand{\phone}{06 xx xx xx xx}
\renewcommand{\email}{xxxxxxxxxx@xxxx.xxx}
\renewcommand{\adress}{xx xxx xxxxxxxxx}
\renewcommand{\city}{xxxxx xxxxxxx xxx xxxxx}
\renewcommand{\profession}{Programmeur}

\linespread{1.2}

\begin{document}

\begin{header}

	% \photo[shadow=true]{photo/photo.png}

	\begin{contact}[vcard=false]
		\addinfo{Email}{\href{mailto:\email}{\email}}
		\addinfo{Téléphone}{\phone}
		%% \addinfo{Adresse}{\adress \\ \city}
    \addinfo{Site web}{\href{https://elaye.github.io}{elaye.github.io}}
		% \addinfo{}{\city}
	\end{contact}

\end{header}

\begin{body}
	\experience
		\begin{entry}
			{2015 - 2017}{Développeur dans une agence digitale}{Londres}
			{Stink Studios}
      \textbf{Projet :} \href{http://www.stinkstudios.com/news/say-hello-to-rita-our-real-time-cloud-rendering-platform}{RITA (Rendering In The Air)}\\
      ~~~~Logiciel de rendu de vidéos dynamiques en temps réel\\
      \textbf{Projet :} visite virtuelle des studios d'Abbey Road - Google Cardboard : \href{https://www.stinkstudios.com/work/google-inside-abbey-road-cardboard}{Inside Abbey Road}\\
      ~~~~Développement Unity\\
      \textbf{Projet :} visite interactive des parcs nationaux des Etats-Unis : \href{http://g.co/nationalparks}{National Parks}\\
			\textbf{Projet :} site web interactif pour le Google Cultural Institute : \href{https://performingarts.withgoogle.com/fr}{Google Performing Arts} \\
			~~~~Développement d'un lecteur vidéo à 360$^{\circ}$
		\end{entry}
		\begin{entry}
			{Mars-Juillet 2014}{Stage dans un laboratoire d'informatique}{Bordeaux}
			{LaBRI - Laboratoire Bordelais de Recherche en Informatique}
			\textbf{Projet :} Création de résumés vidéo scalables de documents culturels en utilisant\\
			un cube de données (projet international franco-mexicain)\\
			~~~~Extraction et clustering de descripteurs de caractéristiques audiovisuels bas niveau\\
			~~~~Programmation en C++
		\end{entry}
		\begin{entry}
			{Juin-Septembre 2013}{Stage en tant qu'assistant ingénieur en \\traitement du signal}{Grenoble}
			{Movea}
			\textbf{Projet :} Détection de pas pour la navigation piétonne\\
			~~~~Analyse de données enregistrées par une centrale inertielle embarquée\\
			~~~~Modèle de détection de pas réalisé avec Simulink
		\end{entry}

	\education
		\begin{entry}
			{2011 - 2014}{Ecole d'ingénieur à Grenoble INP Phelma}{Grenoble}
			{PHELMA - Ecolde d'ingénieur en Physique, Electronique et Matériaux}
			\textit{2013-2014} Semestre Art, Science et Techonologie\\
			\textit{HMI - Interfaces haptiques - Programmation pour la création interactivve}\\
			\textbf{Projet :} Jonglage interactif avec un diabolo\\
      ~~~~Tracking du jongleur avec une Kinect et génération audiovisuelle
			%% ~~~~Tracking des bras du jongleur et du diabolo avec une Kinect\\
			%% ~~~~Visualisation utilisant un système de particules qui réagit aux mouvements du jongleur\\
			%% ~~~~Génération sonore réalisée avec SuperCollider\\
			%% ~~~~{\itshape [~code disponible sur \href{https://github.com/elaye/AST\_diabolo\_particles}{https://github.com/elaye/AST\_diabolo\_particles}~]}
			\medskip\\

			\textit{2012-2013} Physique et nanosciences\\
			\textit{Physique quantique - Physique du solide - Physique des lasers}\\
			\textbf{Projet :} Mémoires MRAM ultrarapides\\
			~~~~Simulation du comportement de nanopilliers magnétiques
			\medskip\\

			% \textit{2011-2012} Physique, Electronique et Télécoms\\
			% \textit{Physique - Electronique - Traitement du signal}\\
			% \textbf{Projet :} Compétition CanSat\\
			% ~~~~Electronique embarquée et programmation Arduino
		\end{entry}
		\begin{entry}
			{2008 - 2011}{Classe Préparatoire aux Grandes Ecoles}{Poitiers}
			{MPSI - MP - MP*}
			Physique et Mathématiques\\
			Option Sciences de l'Ingénieur
		\end{entry}

\end{body}

\begin{aside}

	\begin{about}
		\addinfo{Age:}{27}
		\addinfo{Français:}{langue maternelle}
		\addinfo{Anglais:}{courant}
    \addinfo{Espagnol:}{basique}
		\addinfo{\faGithub}{\href{https://github.com/elaye}{github.com/elaye}}
		\addinfo{\faGlobe}{\href{https://elaye.github.io}{elaye.github.io}}
	\end{about}\noindent

	\skills
	
	\begin{skillgroup}{Langages}
		\skill{C/C++}{3}
    \skill{Haskell}{2}
    \skill{Rust}{1}
	\end{skillgroup}

	\begin{skillgroup}{Web}
		\skill{HTML}{4}
		\skill{CSS}{4}
		\skill{Javascript}{4}
	\end{skillgroup}

	\begin{skillgroup}{OS}
		\skill{GNU/Linux}{3}
		\skill{macOS}{3}
		\skill{Windows}{2}
	\end{skillgroup}

  \begin{skillgroup}{Hardware}
    \skill{Arduino}{3}
    \skill{Raspberry Pi}{3}
  \end{skillgroup}

	\begin{skillgroup}{Divers}
    \skill{OpenGL}{3}
    \skill{GLSL}{3}
		\skill{openFrameworks}{4}
		\skill{\LaTeX}{3}
	\end{skillgroup}

	\skillscaption
\end{aside}

\end{document}
