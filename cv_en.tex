%!TEX program = xelatex

% Pass 'debug' option to show boxes
\documentclass[a4paper]{genard-cv}
\usepackage[english]{babel}

% Information
\renewcommand{\firstname}{Elie}
\renewcommand{\lastname}{Génard}
\renewcommand{\subtitle}{Resume for a developer position}
\renewcommand{\phone}{07xxxxxxxxxxx}
\renewcommand{\email}{xxxxxxxxxxx@xxxxxxx.xxx}
\renewcommand{\adress}{xxxxxxxxxxxxxxx}
\renewcommand{\city}{xxxxxxxxxxxxxxxx}
\renewcommand{\profession}{Developer}

\linespread{1.2}

\begin{document}

\begin{header}

	% \photo[shadow=true]{photo/photo.png}

	\begin{contact}[vcard=false]
		\addinfo{Email}{\href{mailto:\email}{\email}}
		\addinfo{Phone}{\phone}
		\addinfo{Address}{\adress \\ \city}
		\addinfo{Website}{\href{https://elaye.github.io}{elaye.github.io}}
		% \addinfo{}{\city}
	\end{contact}

\end{header}

\begin{body}
	\experience
		\begin{entry}
			{2015 - 2017}{Developer in a studio}{London}
			{Stink Studios}
      \textbf{Project :} \href{http://www.stinkstudios.com/news/say-hello-to-rita-our-real-time-cloud-rendering-platform}{RITA (Rendering In The Air)}\\
      ~~~~Real-time rendering of dynamic videos\\
      \textbf{Projet :} virtual tour of Abbey Road studios - Google Cardboard : \href{https://www.stinkstudios.com/work/google-inside-abbey-road-cardboard}{Inside Abbey Road}\\
      ~~~~Unity development\\
      \textbf{Project :} interactive tour of the National Parks of the United State: \href{http://g.co/nationalparks}{National Parks}\\
			\textbf{Project :} interactive website for the Google Cultural Institute : \href{https://performingarts.withgoogle.com/fr}{Google Performing Arts} \\
			~~~~Development of a 360$^{\circ}$ video player
		\end{entry}
		\begin{entry}
			{Mars-July 2014}{Internship in a computer science laboratory}{Bordeaux}
			{LaBRI - Laboratoire Bordelais de Recherche en Informatique}
			\textbf{Project:} Scalable video summarization of cultural video documents in cross-media space based\\
			 on data cube approach (international French-Mexican project)\\
			~~~~Extraction and clustering of low-level audiovisual feature descriptors\\
			~~~~Software development in C++
		\end{entry}
		\begin{entry}
			{June-September 2013}{Internship as an engineer assistant in signal processing}{Grenoble}
			{Movea}
			\textbf{Project:} Step detection for pedestrian navigation\\
			~~~~Data analysis of embedded inertial measurement unit recordings\\
			~~~~Step detection model made with Simulink
		\end{entry}

	\education
		\begin{entry}
			{2011 - 2014}{Master's degree at Grenoble INP Phelma}{Grenoble}
			{PHELMA - School of Engineering in Physics, Electronics and Materials Science}
			\textit{2013-2014} Art, Science and Techonology semester\\
			\textit{HMI - Haptic interfaces - Programming for interactive creation}\\
			\textbf{Project:} Interactive juggling with a diabolo\\
      ~~~~Tracking of the juggler with a Kinect and audiovisual generation
			%% ~~~~Tracking of the juggler arms and diabolo with a Kinect\\
			%% ~~~~Visual feedback with a particle engine that reacts to juggler movements\\
			%% ~~~~Audio feedback made with SuperCollider\\
			%% ~~~~{\itshape [~More details at \href{https://elaye.github.io/openframeworks/2014/10/01/interactive-diabolo.html}{elaye.github.io/openframeworks/2014/10/01/interactive-diabolo.html}~]}
			\medskip\\

			\textit{2012-2013} Physics and nanosciences\\
			\textit{Quantum physics - Solid-state physics - Laser physics}\\
			\textbf{Project:} Ultrafast MRAM memories\\
			~~~~Simulation of the behaviour of magnetic nanopillars (C++)
			\medskip\\

			%% \textit{2011-2012} Physics, Electronics and Telecom\\
			%% \textit{Physics - Electronics - Signal processing}\\
			%% \textbf{Project:} CanSat competition organized by the CNES\\
			%% ~~~~Embedded electronics and Arduino programming
		\end{entry}
		\begin{entry}
			{2008 - 2011}{Preparatory Class to Grandes Ecoles}{Poitiers}
			{Intensive three-year course to prepare for the competitive entrance
			into France’s leading colleges - MPSI - MP - MP* at Lycée Camille Guérin}
			Physics and mathematics - Engineering science option
		\end{entry}
		% \begin{entry}
		% 	{2008}{High school diploma - Scientific option}{Barbezieux}
		% 	{With honors}
		% \end{entry}

\end{body}

\begin{aside}

	\begin{about}
		\addinfo{Age:}{27}
		\addinfo{French:}{mother tongue}
		\addinfo{English:}{fluent}
		\addinfo{\faGithub}{\href{https://github.com/elaye}{github.com/elaye}}
		\addinfo{\faGlobe}{\href{https://elaye.github.io}{elaye.github.io}}
	\end{about}\noindent

	\skills
	
	\begin{skillgroup}{System}
		\skill{C/C++}{3}
    \skill{Haskell}{2}
    \skill{Rust}{1}
	\end{skillgroup}

	\begin{skillgroup}{Web}
		\skill{HTML}{4}
		\skill{CSS}{4}
		\skill{Javascript}{4}
	\end{skillgroup}

	\begin{skillgroup}{OS}
		\skill{GNU/Linux}{3}
		\skill{macOS}{3}
		\skill{Windows}{2}
	\end{skillgroup}

  \begin{skillgroup}{Hardware}
    \skill{Arduino}{3}
    \skill{Raspberry Pi}{3}
  \end{skillgroup}

	\begin{skillgroup}{Misc}
    \skill{OpenGL}{3}
    \skill{GLSL}{3}
		\skill{openFrameworks}{4}
		\skill{\LaTeX}{3}
	\end{skillgroup}

	\skillscaption
\end{aside}

\end{document}
